% --------------------------
% 中文字体
% --------------------------
\usepackage{xeCJK}
% 中文默认字体:宋体,粗体以黑体代替,斜体以楷书代替
\setCJKmainfont   { SimSun } [ BoldFont = SimHei , ItalicFont = KaiTi ]
% 中文无衬线字体:微软雅黑,粗体为对应的微软雅黑粗体
\setCJKsansfont   { Microsoft~YaHei } [ BoldFont = *~Bold ]
% 中文等宽字体:仿宋
\setCJKmonofont   { FangSong }

% --------------------------
% 英文字体
% --------------------------
\usepackage{fontspec}
\setmainfont{Times New Roman}
\setmonofont{Consolas}

% --------------------------
% 图片支持
% --------------------------
\usepackage{graphicx}
\usepackage{float}

% --------------------------
% 页眉页脚
% --------------------------
\usepackage{fancyhdr}
\fancypagestyle{plain}{
  \fancyhf{}
  \fancyhead[L]{同城配送管理系统}
  \fancyfoot[C]{\thepage}
  \renewcommand{\headrulewidth}{0.4pt}
  \renewcommand{\footrulewidth}{0pt}
}

\pagestyle{fancy}
\fancyhf{}
\fancyhead[L]{同城配送管理系统}
\fancyfoot[C]{第~\thepage~页}

% --------------------------
% 封面页不显示页码
% --------------------------
\usepackage{etoolbox}
\AtBeginDocument{\thispagestyle{empty}}

% ---------------------------
% 层级标号设置(listings)
% ---------------------------
\usepackage{enumitem}
% \setlist{nosep}
\setlistdepth{4}
\setlist[itemize,1]{label=\textbullet}
\setlist[itemize,2]{label=\textopenbullet}
\setlist[itemize,3]{label=\(\triangleright\)}

% ---------------------------
% 代码块背景设置(listings)
% ---------------------------
\usepackage{listings}
\usepackage{xcolor}

% 定义代码框背景色
\definecolor{codebg}{RGB}{240,240,240}  % 浅灰色背景

\lstset{
    backgroundcolor=\color{codebg},      % 背景色
    basicstyle=\ttfamily\small,          % 字体与字号
    breaklines=true,                     % 自动换行
    frame=single,                        % 框线
    frameround=tttt,                     % 四角圆角
    numbers=left,                        % 行号位置
    numberstyle=\tiny\color{gray},       % 行号样式
    keywordstyle=\color{blue},           % 关键字颜色
    commentstyle=\color{green!60!black}, % 注释颜色
    stringstyle=\color{red!60!black},    % 字符串颜色
    showstringspaces=false                % 不显示空格下划线
}

% \usepackage{xcolor}    % 为了渲染颜色,需要使用 xcolor 包
% \usepackage{listings}  % 渲染代码块
%
% % 定义颜色
% \definecolor{codegreen}{rgb}{0,0.6,0}
% \definecolor{codegray}{rgb}{0.5,0.5,0.5}
% \definecolor{codepurple}{rgb}{0.58,0,0.82}
% \definecolor{backcolour}{rgb}{0.95,0.95,0.92}
%
% % 定义 listings 风格,可以定义多个
% \lstdefinestyle{mystyle}{
%     backgroundcolor=\color{backcolour},  % 背景色
%     commentstyle= \color{red!50!green!50!blue!50},  % 注释的颜色
%     keywordstyle= \color{blue!70},  % 关键字/程序语言中的保留字颜色
%     numberstyle=\tiny\color{codegray},  % 左侧行号显示的颜色
%     stringstyle=\color{codepurple},
%     basicstyle=\ttfamily\footnotesize,
%     breakatwhitespace=false,
%     breaklines=true,  % 对过长的代码自动换行
%     captionpos=b,
%     keepspaces=true,
%     numbers=left,  % 在左侧显示行号
%     numbersep=5pt,
%     showspaces=false,
%     showstringspaces=false,  % 不显示字符串中的空格下划线
%     showtabs=false,
%     tabsize=2,
%     frame=single  % [none | single | shadowbox] 显示边框
% }
%
% \lstset{style=mystyle}  % 使用 listings 风格

